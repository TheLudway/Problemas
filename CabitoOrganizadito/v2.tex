\documentclass[12pt,a4paper]{article}

% --- Packages ---
\usepackage[utf8]{inputenc}
\usepackage[T1]{fontenc}
\usepackage{geometry}
\geometry{margin=2.5cm}
\usepackage{array}
\usepackage{fancyhdr}
\usepackage{titlesec}
\usepackage{tcolorbox}   % for nice boxes
\tcbuselibrary{listings,skins,breakable}
\usepackage{amsmath,amssymb}
\usepackage{float}

% --- Formatting ---
\pagestyle{fancy}
\fancyhf{}
% Increase header height to avoid overlap
\setlength{\headheight}{40pt}


% Move footer up (default ~30pt)
\setlength{\footskip}{20pt}

% Left header (university and exam info)
\lhead{%
  \parbox[t]{0.48\textwidth}{
    \small
    Universidad Jorge Tadeo Lozano\\
    Estructuras de Datos\\
    Prueba de Programación: Segundo momento 2025-2 (\textbf{Hasta Diccionarios y Listas})%
  }%
}

% Right header (date, professor, monitor)
\rhead{%
  \parbox[t]{0.48\textwidth}{
    \small
    \textbf{Fecha}: 17 de septiembre de 2025\\
    \textbf{Profesor}: José Alejandro Franco Calderón\\
    \textbf{Monitor}: Ludwig Alvarado Becerra%
  }%
}

% Optional: page number in center footer
\cfoot{\thepage}

% Paragraph spacing
\setlength{\parskip}{1em}

\titleformat{\section}{\large\bfseries}{\thesection.}{0.5em}{}

% --- Environment for samples ---
\newenvironment{sample}[2]{
  \begin{tcolorbox}[title=Prueba \##1, colback=white, colframe=black, sharp corners,
                    fonttitle=\bfseries, breakable]
  \begin{tabular}{|p{0.45\linewidth}|p{0.45\linewidth}|}
    \hline
    \textbf{Entrada} & \textbf{Salida} \\
    \hline
}{\\ \hline
  \end{tabular}
  \end{tcolorbox}
}

% --- Document ---
\begin{document}


\begin{center}
    {\LARGE \bf Cabito Organizadito\textsuperscript{*}}\\[1ex]
    Autor: Ludwig Alvarado
\end{center}


Cabito (figura \ref{fig:cabos-2} a la izquierda) es la mascota de la Universidad Jorge Tadeo Lozano. Él es el sucesor de Cabo (figura \ref{fig:cabos-2} a la derecha, lo vamos a llamar Don Cabo), un perrito muy querido por toda la comunidad Tadeísta (al igual que Cabito) hace muchos años\ldots Cabito llego como un compañero de Don Cabo y es el que hoy en día conocemos en la Tadeo. Estos dos amados perritos en el corto tiempo que estuvieron juntos se pasaron algunas \textit{mañas}, en especial Don Cabo (como gran mentor) a Cabito.


\begin{figure}[H]
  \centering
  \includegraphics[width=0.6\textwidth]{img/img1.jpg}
  \caption{Cabito y Cabo. Autor de la foto: Universidad Jorge Tadeo Lozano}
  \label{fig:cabos-2}
\end{figure}


Entre una de las muchas \textit{mañas}, está la comida\ldots Don Cabo en un día de enseñanzas a Cabito le dijo: ``\textit{guau guau, guauuu guau guau woof, woof guau woooof\ldots}'' En español esto quiere decir ``Cabito, cuando tengas hambre, pídele comida a esos ingenuos estudiantes\ldots'' Siguiendo sus aprendizajes, Cabito mira con sus tiernos ojos a los estudiantes en busca de comida y ellos le dan (\textbf{¡Nunca le debes dar comida a Cabito!}).

Cada vez que alguien le da comida a Cabito él se va a su \textit{guarida secreta} y coloca la comida que le han dado en el suelo y organiza las unidades de comida por filas. Cabito sigue el siguiente patrón; en la primera fila coloca 1 unidad de comida, en la segunda fila coloca 2 unidades de comida, y así sucesivamente, entonces en la fila \(i\) Cabito coloca \(i\) unidades de comida.

Al final de cada día Cabito quiere saber cuántas filas de comida puede organizar y tu misión es escribir un programa para ayudarlo.

\subsection*{Entrada del programa}


Cabito te va a dar un número entero (\(1 \leq n \leq 100\)), la cantidad total de unidades de comida que logró recolectar a lo largo del día. Recuerda que Cabito organiza su comida por filas y en la primera fila debe haber 1 unidad de comida, en la segunda fila 2 unidades de comida, en la tercera fila 3 unidades de comida y así sucesivamente\ldots


\subsection*{Salida del programa}

Lo que tu programa debe hacer es imprimir un número entero, la cantidad de filas que Cabito puede formar con el \(n\) dado. Si Cabito te da \(n = 3\) (tres unidades de comida), en la primera fila hay una unidad de comida y en la segunda dos. Por lo tanto, lo que debes imprimir es \texttt{2} (porque hay 2 filas en total). Sin embargo, hay un detalle extra, en la fila \(i\) \textbf{deben haber máximo} \(i\) unidades de comida. Es decir, en el caso de que Cabito te de \(n=4\) la comida se debe organizar así; en la primera fila una unidad de comida, en la segunda fila 2 unidades de comida y en la tercera fila 1 unidad de comida. Por lo tanto, lo que debe imprimir el programa es \texttt{3}.

\subsection*{Casos de prueba de ejemplo}

\begin{sample}{1}{}
\texttt{3}
&
\texttt{2}
\end{sample}

\begin{sample}{2}{}
\texttt{6}
&
\texttt{3}
\end{sample}

\begin{sample}{3}{}
\texttt{4}
&
\texttt{3}
\end{sample}



\subsection*{Notas}

En la primera prueba, Cabito reparte \texttt{3} unidades: pone 1 en la primera fila y 2 en la segunda, usando en total \texttt{2} filas.

En la tercera prueba, con \texttt{4} unidades: coloca 1 en la primera fila y 3 en la segunda. No alcanza para una tercera (que requeriría 6 unidades en total, vea prueba 2), así que también se usan \texttt{2} filas.

\subsection*{Aspectos Importantes}

\begin{enumerate}
  \item La prueba se resolverá en \textbf{2 momentos}: primero en papel, sin editor de código (\textbf{valdrá 3.0 puntos}). Segundo, pasará el algoritmo desarrollando en papel al editor de código (\textbf{valdrá 2.0 puntos}).
  \item Hagan uso de los conceptos aprendidos hasta el tema de \textbf{diccionarios y listas}.
  \item Asuman que el valor que se va a ingresar es de tipo entero positivo.
  \item Pueden hacer uso del concepto de listas por comprensión.
  \item Pueden usar solo la función \texttt{range()}, no se permite el uso de otras funciones, si tienen dudas pregunten al profesor si es permitida, de lo contrario si la usa y no es permitida (-1.0 puntos).
  \item No importe ninguna librerías externas, no las necesita, en caso de hacerlo (-1.0 puntos).
  \item Si utilizan declaraciones \texttt{break} o \texttt{continue} (-0.2 puntos).
  \item No preocuparse por validaciones de errores, asuma que los datos se suministran de manera correcta.
  \item Prestar atención al formato de entrada como al formato de salida para la escritura y presentación del algoritmo.
\end{enumerate}


\end{document}
